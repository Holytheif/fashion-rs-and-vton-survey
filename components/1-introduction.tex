\section{Introduction} \label{section:intro}
	The fashion industry has undergone a profound transformation in recent years, largely propelled by advancements in artificial intelligence (AI), computer vision, and augmented reality (AR). The integration of these technologies has given rise to innovative solutions in the realms of virtual try-on systems, reshaping the way consumers explore, select, and interact with clothing and accessories. In today's fast-paced digital world, where e-commerce reigns supreme, these systems have become indispensable tools for both retailers and consumers alike.

	This survey paper explores the dynamic landscape of clothing recommendation and virtual try-on systems, shedding light on their evolving capabilities and significant impact on the e-commerce ecosystem. In this exploration of the state of these technologies, we delve into the underlying methodologies, recent breakthroughs, and the multitude of applications across various e-commerce platforms. As we navigate this survey, we uncover the pivotal role these systems play in shaping the future of fashion retail, fostering more informed and immersive shopping experiences for consumers worldwide.

	Previous surveys have provided a comprehensive list of techniques and technologies available in the space, going into details of how computer vision was being used for fashion detection, analysis, and synthesis \cite{DBLP:journals/csur/ChengSCHL21, Jain_Wah_2022}, how modern recommender systems work \cite{DBLP:journals/corr/abs-2202-02757, DBLP:journals/sncs/ShirkhaniMSH23}, the state of virtual try-on systems \cite{DBLP:journals/corr/abs-2111-00905, DBLP:journals/mta/GhodhbaniNRA22, DBLP:journals/cvm/LiangL21}, the use, impact, and challenges of augmented reality in the fashion industry \cite{menon2020impact, jayamini2021use, DBLP:journals/corr/abs-2202-09450, huang2019enhancing, mehta2020enhancement, zak2020augmented, caboni2019augmented}, and the general use of AI in fashion design and commerce \cite{DBLP:journals/access/GiriJZB19, DBLP:journals/corr/abs-2105-03050, DBLP:journals/access/GuoZLCCW23, DBLP:journals/spm/ChenSC23, sahni2021review, liang2020implementation, sareen2022ai, 10153335, DBLP:journals/tmm/Yan0LZX0Y23}.

	The rest of the survey is structured as follows: Section \ref{section:rs} is dedicated to recommendation systems, where we discuss the techniques used and provide a comparative analysis of the research in the field. In Section \ref{section:vton}, we explore virtual try-on systems, where we go over the prominent work done in the area. In Section \ref{section:datasets} we list and discuss the various datasets available for use in the domain. And finally, we conclude in Section \ref{section:conclusion}.