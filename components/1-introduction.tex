\section{\textbf{Introduction}} \label{section:intro}
	In today's fast-paced world, the biggest appeal of e-commerce is the ability to view and purchase items from a large catalog without having to physically visit the store. Getting items delivered to the doorstep is convenient because it saves time and mental overhead on many levels. However, while this is true for most items, the fashion industry is still lagging behind.

	The online fashion retail market sees return rates of as high as 50\% with "did not fit", "did not like", and "planned product return (show-rooming)" as some of the main cited reasons \cite{stocker2021new}.

	The two solutions to this problem are good recommendation systems which suggest clothing that look good on the consumer and aligns with their preferences, and virtual try-on systems that eliminate the need to buy a selection of items only to return most of them.

	This survey paper explores the field of clothing recommendation and virtual try-on systems, shedding light on their evolving capabilities and significant impact on the e-commerce ecosystem. In this exploration of the state of these technologies, we delve into the underlying methodologies, recent breakthroughs, and the multitude of applications across various e-commerce platforms. As we navigate this survey, we uncover the pivotal role these systems play in shaping the future of fashion retail, fostering more informed and immersive shopping experiences for consumers worldwide.

	Previous surveys have provided a comprehensive list of techniques and technologies available in the space, going into details of how computer vision was being used for fashion detection, analysis, and synthesis \cite{DBLP:journals/csur/ChengSCHL21, Jain_Wah_2022}, how modern recommender systems work \cite{DBLP:journals/corr/abs-2202-02757, DBLP:journals/sncs/ShirkhaniMSH23}, the state of virtual try-on systems \cite{DBLP:journals/corr/abs-2111-00905, DBLP:journals/mta/GhodhbaniNRA22, DBLP:journals/cvm/LiangL21}, the use, impact, and challenges of augmented reality in the fashion industry \cite{menon2020impact, jayamini2021use, DBLP:journals/corr/abs-2202-09450, huang2019enhancing, mehta2020enhancement, zak2020augmented, caboni2019augmented}, and the general use of AI in fashion design and commerce \cite{DBLP:journals/access/GiriJZB19, DBLP:journals/corr/abs-2105-03050, DBLP:journals/access/GuoZLCCW23, DBLP:journals/spm/ChenSC23, sahni2021review, liang2020implementation, sareen2022ai, 10153335, DBLP:journals/tmm/Yan0LZX0Y23}.

	The objectives of this survey are to:

	\begin{enumerate}
		\item Offer a thorough and up-to-date review of the state-of-the-art in AI-based clothing recommendation and virtual try-on technologies.
		\item Explore and evaluate various technical approaches and methodologies employed in the field, discussing their strengths and limitations.
		\item Discuss real-world applications and case studies of these technologies.
		\item Address the challenges and potential future developments in the field, outlining opportunities for further research and innovation.
		\item Serve as a valuable resource for researchers interested in understanding the intersection of AI, fashion, and e-commerce.
	\end{enumerate}

	This review is structured as follows: Section \ref{section:survey-methodology} describes the methodology behind the survey. Section \ref{section:rs} is dedicated to recommendation systems, where we discuss the techniques used and provide a comparative analysis of the research in the field. In Section \ref{section:vton}, we explore virtual try-on systems, where we go over the prominent work done in the area. In Section \ref{section:datasets} we discuss available datasets and evaluation metrics used in the domain. And finally, we conclude in Sections \ref{section:conclusion} and \ref{section:future}.