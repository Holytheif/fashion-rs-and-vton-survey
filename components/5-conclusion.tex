\section{\textbf{Conclusion}} \label{section:conclusion}
	Throughout this survey, we have explored the nuances and challenges that fashion recommenders and virtual try-on systems present, recognizing that they are not just complementary but need to be developed side-by-side in harmony.

	AI-driven recommendation systems, while still evolving, hold the promise of more personalized, efficient, and satisfying online shopping experiences. The fusion of deep learning, data analysis, and user feedback is making it possible to get closer towards a real digital assistant who can recommend clothing that best suits a person.

	Simultaneously, virtual try-on is advancing steadily, offering consumers the ability to virtually experience clothing before purchasing. These immersive experiences, while not yet ubiquitous, demonstrate immense potential in mitigating buyer uncertainty and reducing return rates. The integration of augmented reality and computer vision continues to evolve, holding the potential to blur the lines between physical and digital fashion retail.

	As we contemplate the future, the co-evolution of recommendation systems and virtual try-on paints an optimistic picture. Their continued development side-by-side holds the promise of a fashion e-commerce landscape where customers enjoy highly personalized, risk-free shopping, and retailers benefit from increased customer loyalty and reduced operational costs. The ultimate result is a win-win scenario where technology bridges the gap between consumer desire and industry innovation, reshaping the fashion e-commerce sphere into a more engaging, efficient, and enjoyable marketplace for all.